% Basic document formatting (must come before bidi)
\usepackage{amsmath}
\usepackage{amsthm}
\usepackage{amssymb}

% Tables and graphics (must come before bidi)
\usepackage[table]{xcolor}
\usepackage{graphicx}
\usepackage{booktabs}
\usepackage{threeparttable}
\usepackage{makecell}
\usepackage{colortbl}
\usepackage{array}

% Formatting packages (must come before bidi)
\usepackage[a4paper,margin=1in,footskip=0.25in]{geometry}
\usepackage{float}
\floatstyle{plain}
\restylefloat{table}
\restylefloat{figure}
\usepackage{enumitem}
\usepackage{caption}
\usepackage{subcaption}
\usepackage{fancybox}
\usepackage{fancyhdr}
\usepackage{adjustbox}
\usepackage{varwidth}

% TikZ and PGF (must come before bidi)
\usepackage{tikz, pgfplots}
\usetikzlibrary{calc}
\usetikzlibrary{arrows,arrows.meta}
\usetikzlibrary{decorations.markings}
\usetikzlibrary{angles,quotes}
\usetikzlibrary{intersections}
\usetikzlibrary{fadings}
\usetikzlibrary{decorations.pathreplacing}
\usetikzlibrary{plotmarks}
\usetikzlibrary{shadows}  % Add shadows library for shadowbox compatibility

% Other packages that must come before bidi
\usepackage{svg}
\usepackage{transparent}
\usepackage{lmodern}

% Math and science packages
\usepackage{siunitx}
\sisetup{
  detect-family,
  separate-uncertainty = true,
  table-number-alignment = center,
  table-figures-integer = 2,
  table-figures-decimal = 2,
  table-figures-uncertainty = 2,
  exponent-product = \times
}

% Hyperref (must come before polyglossia/bidi)
\usepackage{hyperref}
\usepackage{bookmark}

% Load mdframed before bidi
\usepackage[framemethod=TikZ]{mdframed}

% tcolorbox setup (before bidi)
\usepackage[skins, most]{tcolorbox}
\tcbuselibrary{listings,breakable}
\tcbuselibrary{theorems}

% Theorem tools (before bidi)
\usepackage{thmtools}

% Other utilities
\usepackage{smartref}
\usepackage{etexcmds}
\usepackage{xparse}

% Hebrew support (MUST BE LAST - loads bidi automatically)
\usepackage{fontspec}
\usepackage{polyglossia}
\setmainlanguage{hebrew}
\setotherlanguage{english}

% Bibliography (MUST come after polyglossia)
\usepackage[backend=biber,style=numeric,sorting=none,backref=false]{biblatex}
\addbibresource{bibliography.bib}

% Configure biblatex for Hebrew documents with XeLaTeX
\DeclareFieldFormat{postnote}{#1}
\DeclareFieldFormat{multipostnote}{#1}
\renewcommand*{\bibfont}{\small}

% Fix citation command for Hebrew/RTL documents
\DeclareCiteCommand{\cite}
  {\usebibmacro{prenote}}
  {\usebibmacro{citeindex}%
   \printtext[bibhyperref]{[\usebibmacro{cite}]}}
  {\multicitedelim}
  {\usebibmacro{postnote}}

% Ensure proper URL handling
\setcounter{biburllcpenalty}{9000}
\setcounter{biburlucpenalty}{9000}

% Configure bibliography list for Hebrew
\defbibenvironment{bibliography}
  {\list
     {\printtext[labelnumberwidth]{%
        \printfield{labelprefix}%
        \printfield{labelnumber}}}
     {\setlength{\labelwidth}{\labelnumberwidth}%
      \setlength{\leftmargin}{\labelwidth}%
      \setlength{\labelsep}{\biblabelsep}%
      \addtolength{\leftmargin}{\labelsep}%
      \setlength{\itemsep}{\bibitemsep}%
      \setlength{\parsep}{\bibparsep}}%
      \renewcommand*{\makelabel}[1]{\hss##1}}
  {\endlist}
  {\item}

% Fix brackets format for Hebrew RTL
\DeclareFieldFormat{labelnumberwidth}{\mkbibbrackets{#1}}
\renewcommand*{\mkbibbrackets}[1]{[#1]}

% High-quality fonts with superior Latin/punctuation support
\setmainfont{FreeSerif}[
  Ligatures=TeX,
  Scale=1.0
]

\setsansfont{DejaVu Sans}[
  Ligatures=TeX,
  Scale=MatchLowercase
]

\setmonofont{DejaVu Sans Mono}[
  Scale=MatchLowercase
]

% David CLM Hebrew fonts - professional Israeli academic typography
\newfontfamily\hebrewfont{David CLM}[
  Script=Hebrew,
  Scale=1.0
]

\newfontfamily\hebrewfontsf{David CLM}[
  Script=Hebrew,
  Scale=MatchLowercase
]

\newfontfamily\hebrewfonttt{DejaVu Sans Mono}[
  Script=Hebrew,
  Scale=MatchLowercase
]

% Overleaf-style font optimization
\defaultfontfeatures{Ligatures=TeX,Scale=MatchLowercase}
\defaultfontfeatures[\rmfamily]{Ligatures=TeX,Scale=1}

% Note: bidi package is automatically loaded by polyglossia

% tcolorbox styles
\tcbset {
  base/.style={
    arc=0mm, 
    bottomtitle=0.5mm,
    boxrule=0mm,
    colbacktitle=black!10!white, 
    coltitle=black, 
    fonttitle=\bfseries, 
    right=2.5mm,
    rightrule=1mm,
    left=3.5mm,
    title={#1},
    toptitle=0.75mm, 
  }
}

\newtcolorbox{mybox}[2][]{enhanced,colback=white,colframe=green!65!black,
enlarge top by=10mm,
title={\textbf{#2}}, coltitle=green!65!black, colbacktitle=red!65!black, 
overlay={ \path[fill=blue!65,line width=.4mm] (frame.north west)--++(17mm,0)coordinate(n2)--++(0,8mm)--++(-20mm,0) arc (-90:90:-4mm)--cycle;
\node at ([shift={(5mm,4mm)}]frame.north west){\color{white}{\textbf{\sffamily EXAMPLE}}}}
#1}

% GRAPHS
% graph xy plot, maximum of 3 functions
\def\maxFunc{3}

\NewDocumentCommand{\plotxy}{O{0} O{0} O{0}}{%
\begin{tikzpicture}[>=latex]
    \begin{axis}[
      axis x line=center,
      axis y line=center,
      xtick={-20,-19,...,20},
      ytick={-20,-19,...,20},
      xlabel={$x$},
      ylabel={$y$},
      xlabel style={below right},
      ylabel style={above left},
      xmin=-5.5,
      xmax=5.5,
      ymin=-5.5,
      ymax=5.5]
      \addplot [mark=none,domain=-4:4] {#1};
      \addplot [mark=none,domain=-4:4] {#2};
      \addplot [mark=none,domain=-4:4] {#3};
    \end{axis}
    \end{tikzpicture}
}

% Theorems
\makeatother

\declaretheoremstyle[
	headfont=\bfseries\sffamily\color{gray!70!black}, bodyfont=\normalfont,
	mdframed={
			linewidth=2pt,
			leftline=false, topline=false, bottomline=false,
			linecolor=ForestGreen, backgroundcolor=ForestGreen!5,
			nobreak=false
		}
]{thmgreenbox}

\declaretheoremstyle[
	headfont=\bfseries\sffamily\color{ForestGreen!70!black}, bodyfont=\normalfont,
	mdframed={
			linewidth=2pt,
			leftline=false, topline=false, bottomline=false,
			linecolor=ForestGreen, backgroundcolor=ForestGreen!8,
			nobreak=false
		}
]{thmgreen2box}

\declaretheoremstyle[
	headfont=\bfseries\sffamily\color{WildStrawberry!70!black}, bodyfont=\normalfont,
	mdframed={
			linewidth=2pt,
			leftline=false, topline=false, bottomline=false,
			linecolor=WildStrawberry, backgroundcolor=WildStrawberry!5,
			nobreak=false
		}
]{thmpinkbox}

\declaretheorem[style=thmgreenbox, name=הגדרה, numberwithin=section]{definition}
\declaretheorem[style=thmgreen2box, name=הגדרה, numbered=no]{definition*}
\declaretheorem[style=thmpinkbox, name=תרגיל, numberwithin=section]{problem}
\declaretheorem[style=thmpinkbox, name=תרגיל, numbered=no]{problem*}

\newcommand{\parinfobox}[1]{
\left{
        \tcbox[colframe=green!50!white,colback=white,width=1pt]{ %adjust the width corresponding to your right margin
            \begin{minipage}{0.35\linewidth}#1\end{minipage} %same goes for this width
      }  
    }
}

\newcommand{\textHebrew}[1]{\text{\texthebrew{#1}}}
\renewcommand{\labelitemi}{$\bullet$}

\tikzset{>=latex} % for LaTeX arrow head

\colorlet{myblue}{blue!80!black}
\colorlet{mydarkblue}{blue!35!black}
\colorlet{myred}{black!50!red}
\colorlet{glasscol}{blue!10}
\colorlet{Ecol}{orange!90!black}
\tikzstyle{myarr}=[-{Latex[length=3,width=2]}]
\tikzstyle{Evec}=[Ecol,{Latex[length=2.8,width=2.5]}-{Latex[length=2.8,width=2.5]},line width=0.6]
\tikzstyle{glass}=[top color=glasscol!88!black,bottom color=glasscol,shading angle=0]
%\tikzstyle{glass}=[top color=glasscol!88!black,bottom color=glasscol,middle color=glasscol!98!black,shading angle=0]
\tikzset{
  light beam/.style={thick,red,decoration={markings,
                     mark=at position #1 with {\arrow{latex}}},
                     postaction={decorate}},
  light beam/.default=0.5}

\newcommand\rightAngle[4]{
  \pgfmathanglebetweenpoints{\pgfpointanchor{#2}{center}}{\pgfpointanchor{#3}{center}}
  \coordinate (tmpRA) at ($(#2)+(\pgfmathresult+45:#4)$);
  \draw[white,line width=0.6] ($(#2)!(tmpRA)!(#1)$) -- (tmpRA) -- ($(#2)!(tmpRA)!(#3)$);
  \draw[blue!40!black] ($(#2)!(tmpRA)!(#1)$) -- (tmpRA) -- ($(#2)!(tmpRA)!(#3)$);
}

% WAVEFRONT
\def\p{0.03}
\def\r{0.25}
\tikzset{
  wavefront/.pic={
    \tikzset{/wavefront/.cd,#1}
    \fill (0,0) circle (\p);
    \draw (\wang:\r) arc(\wang:-\wang:\r);
  }
  /wavefront/.search also={/tikz},
  /wavefront/.cd,
  ang/.store in=\wang, ang={60},
}

\colorlet{xcol}{blue!60!black}
\colorlet{myred}{red!85!black}
\colorlet{myblue}{blue!80!black}
\colorlet{mycyan}{cyan!80!black}
\colorlet{mygreen}{green!70!black}
\colorlet{myorange}{orange!90!black!80}
\colorlet{mypurple}{red!50!blue!90!black!80}
\colorlet{mydarkred}{myred!80!black}
\colorlet{mydarkblue}{myblue!80!black}
\tikzstyle{xline}=[xcol,thick]
\tikzstyle{Tline}=[line width=0.6]
\tikzstyle{width}=[{Latex[length=5,width=3]}-{Latex[length=5,width=3]},thick]
\def\tick#1#2{\draw[thick] (#1)++(#2:0.12) --++ (#2-180:0.24)}
\def\N{100} % number of samples

\colorlet{myblue}{blue!80!black}
\colorlet{myred}{black!50!red}
\colorlet{watercol}{blue!70!cyan!50}
\tikzstyle{myarr}=[-{Latex[length=3,width=2]}]
\tikzstyle{water}=[ball color=watercol]
\tikzset{
  beam/.style={very thick,line cap=round,line join=round},
}

\newcommand\degree{^\circ}